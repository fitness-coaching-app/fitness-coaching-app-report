\section{การออกแบบหลักการคำนวณคะแนนความถูกต้องของท่าทางผู้ใช้}
การคำนวณคะแนนท่าทางของผู้ใช้ จะออกมาในรูปแบบเปอร์เซ็นความถูกต้องเฉลี่ยของท่าทางทั้งหมด ซึ่งจะหาได้จากสมการดังนี้ โดยกำหนดให้ $n$ คือจำนวนท่าทางในคอร์สทั้งหมด
\begin{equation}
    score_{course} = \frac{\sum_{i=1}^{n}{score_{pose_i}}}{n}
\end{equation}
และในแต่ละท่าทางการออกกำลังกาย จะสามารถคำนวณความถูกต้องได้จากเกณฑ์ต่าง ๆ ที่ตรวจสอบ ซึ่งถ้ามีเกณฑ์ใด ๆ จากท่าทางย่อยที่ไม่ตรงตามเกณฑ์ จะถือว่าท่าทางย่อยนั้นไม่ถูกต้อง ซึ่งการคำนวณเปอร์เซ็นต์ความถูกต้องจะนำท่าทางย่อยที่ถูกต้องหารกับท่าทางย่อยที่ได้ทำทั้งหมดซึ่งจะรวมไปถึงท่าทางที่ผิด จะเขียนเป็นสมการได้ ดังนี้
\begin{equation}
    score_{exercise pose} = \frac{correct subpose}{all subpose}
\end{equation}