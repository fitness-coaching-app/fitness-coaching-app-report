\section{การออกแบบหลักการคำนวณคะแนนความถูกต้องของท่าทางผู้ใช้}
\subsection{การคำนวณคะแนนของผู้ใช้ ในกรณีที่เป็นการนับจำนวนท่าทาง}
การคำนวณคะแนนท่าทางของผู้ใช้สำหรับกรณีที่เป็นการนับจำนวนท่าทาง จะสามารถคำนวณความถูกต้องได้จากเกณฑ์ต่าง ๆ ที่ตรวจสอบ ซึ่งถ้ามีเกณฑ์ใด ๆ จากท่าทางย่อยที่ไม่ตรงตามเกณฑ์ จะถือว่าท่าทางย่อยนั้นไม่ถูกต้อง ซึ่งการคำนวณเปอร์เซ็นต์ความถูกต้องจะนำท่าทางย่อยที่ถูกต้องหารกับท่าทางย่อยที่ได้ทำทั้งหมดซึ่งจะรวมไปถึงท่าทางที่ผิด จะเขียนเป็นสมการได้ ดังนี้
\begin{equation}
    score_{exercise pose} = \frac{correct subpose}{all subpose}
\end{equation}

\subsection{การคำนวณคะแนนของผู้ใช้ ในกรณีที่เป็นการจับเวลา}
การคำนวณคะแนนท่าทางของผู้ใช้สำหรับกรณีที่เป็นการจับเวลา จะเป็นการคำนวณเปอร์เซ็นต์เวลาที่ผู้ใช้ทำท่าทางถูกต้อง เมื่อเทียบกับเวลาทั้งหมดที่ผู้ใช้ได้ออกกำลังกายจริง ตัวอย่างเช่น หากผู้ใช้ได้ทำท่าทางไม่ถูกต้องเป็นเวลา 2 วินาที เวลาหลักที่ใช้นับการทำท่าทางของทั้งคอร์สจะหยุดลงชั่วคราวจนกว่าผู้ใช้จะทำท่าทางได้ถูกต้อง และในขณะเดียวกันจะมีการนับเวลาที่ผู้ใช้ได้ทำท่าทางผิดพลาด โดยจะเขียนเป็นสมการได้ ดังนี้
\begin{equation}
     score_{exercise pose} = \frac{time_{all} - time_{incorrect}}{time_{all}}
\end{equation}


\subsection{การคำนวณคะแนนของผู้ใช้ของทั้งคอร์ส}
การคำนวณคะแนนท่าทางของผู้ใช้สำหรับทั้งคอร์ส จะออกมาในรูปแบบเปอร์เซ็นความถูกต้องเฉลี่ยของท่าทางทั้งหมด ซึ่งจะหาได้จากสมการดังนี้ โดยกำหนดให้ $n$ คือจำนวนท่าทางในคอร์สทั้งหมด
\begin{equation}
    score_{course} = \frac{\sum_{i=1}^{n}{score_{pose_i}}}{n}
\end{equation}

