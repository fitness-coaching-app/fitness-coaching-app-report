\section{การออกแบบหลักการคำนวณคะแนนความถูกต้องของท่าทางผู้ใช้}
\subsection{การคำนวณคะแนนของผู้ใช้ ในกรณีที่เป็นการนับจำนวนท่าทาง}
การคำนวณคะแนนท่าทางของผู้ใช้สำหรับกรณีที่เป็นการนับจำนวนท่าทาง จะสามารถคำนวณความถูกต้องได้จากเกณฑ์ต่าง ๆ ที่ตรวจสอบ ซึ่งถ้ามีเกณฑ์ใด ๆ จากท่าทางย่อยที่ไม่ตรงตามเกณฑ์ จะถือว่าท่าทางย่อยนั้นไม่ถูกต้อง ซึ่งการคำนวณเปอร์เซ็นต์ความถูกต้องจะนำท่าทางย่อยที่ถูกต้องหารกับท่าทางย่อยที่ได้ทำทั้งหมดซึ่งจะรวมไปถึงท่าทางที่ผิด จะเขียนเป็นสมการได้ ดังนี้
\begin{equation}
    score_{exercise pose} = \frac{correct subpose}{all subpose}
\end{equation}

\subsection{การคำนวณคะแนนของผู้ใช้ ในกรณีที่เป็นการจับเวลา}
การคำนวณคะแนนท่าทางของผู้ใช้สำหรับกรณีที่เป็นการจับเวลา จะเป็นการคำนวณเปอร์เซ็นต์เวลาที่ผู้ใช้ทำท่าทางถูกต้อง เมื่อเทียบกับเวลาทั้งหมดที่ผู้ใช้ได้ออกกำลังกายจริง ตัวอย่างเช่น หากผู้ใช้ได้ทำท่าทางไม่ถูกต้องเป็นเวลา 2 วินาที เวลาหลักที่ใช้นับการทำท่าทางของทั้งคอร์สจะหยุดลงชั่วคราวจนกว่าผู้ใช้จะทำท่าทางได้ถูกต้อง และในขณะเดียวกันจะมีการนับเวลาที่ผู้ใช้ได้ทำท่าทางผิดพลาด โดยจะเขียนเป็นสมการได้ ดังนี้
\begin{equation}
     score_{exercise pose} = \frac{time_{all} - time_{incorrect}}{time_{all}}
\end{equation}


\subsection{การคำนวณคะแนนของผู้ใช้ของทั้งคอร์ส}
การคำนวณคะแนนท่าทางของผู้ใช้สำหรับทั้งคอร์ส จะออกมาในรูปแบบเปอร์เซ็นความถูกต้องเฉลี่ยของท่าทางทั้งหมด ซึ่งจะหาได้จากสมการดังนี้ โดยกำหนดให้ $n$ คือจำนวนท่าทางในคอร์สทั้งหมด
\begin{equation}
    score_{course} = \frac{\sum_{i=1}^{n}{score_{exercise pose_i}}}{n}
\end{equation}

\subsection{ตัวอย่างวิธีการคำนวณคะแนนของผู้ใช้}
จากสมการการคำนวณคะแนนท่าทางของผู้ใช้ที่ได้กล่าวมาข้างต้น ขอยกตัวอย่างวิธีการคำนวณคะแนนของผู้ใช้สำหรับกรณีที่เป็นการนับจำนวนท่าทางจากท่าทางกระโดดตบ ซึ่งในท่าทางกระโดดตบนั้นจะมีท่าทางย่อย 2 ท่าทาง คือท่าทางการนำมือลง และท่าทางการยกมือขึ้น เมื่อผู้ใช้ทำท่าทางถูกต้องครบทั้ง 2 ท่าทางย่อย จะถูกนับเป็นการกระโดดตบได้ 1 ครั้ง โดยจากการออกแบบไฟล์การประมวลผลท่าทางของท่าทางกระโดดตบที่ได้กล่าวในหัวข้อที่ผ่านมา ในท่าทางกระโดดตบได้กำหนดให้ผู้ใช้กระโดดตบครบ 10 ครั้งจึงจะเสร็จสิ้น ดังนั้นในการกระโดดตบจำนวน 10 ครั้ง จะมีท่าทางย่อยที่ถูกต้องจำนวนทั้งหมด 20 ท่าทาง โดยตัวอย่างการคำนวณคะแนนจะแบ่งเป็น 2 กรณี ดังนี้ 
\begin{enumerate}
    \item กรณีที่ผู้ใช้ได้ทำท่าทางกระโดดตบโดยที่ทำท่าทางย่อยถูกต้องทั้งหมด หมายความว่าท่าทางย่อยที่ได้ทำทั้งหมดจะเท่ากับ 20 ท่าทางเช่นเดียวกัน คะแนนจึงมีค่าเท่ากับ 100 \% ดังสมการ
    \begin{equation}
        score_{exercise pose} = \frac{correct subpose}{all subpose}
    \end{equation}
    \begin{equation}
        score_{exercise pose} = \frac{20}{20}
    \end{equation}
    \begin{equation}
        score_{exercise pose} = 100 \%
    \end{equation}
    \item กรณีที่ผู้ใช้ทำท่าทางย่อยที่ไม่ถูกต้อง เมื่อผู้ใช้กระโดดตบครบตามจำนวนครั้งที่กำหนดจะทำให้ท่าทางย่อยที่ได้ทำทั้งหมดมีจำนวนเพิ่มขึ้น และส่งผลให้คะแนนมีค่าลดน้อยลงตามจำนวนท่าทางย่อยที่ทำผิดพลาดที่เพิ่มขึ้น เช่นในการกระโดดตบทั้งหมด 10 ครั้ง ในระหว่างการกระโดดตบผู้ใช้ได้ออกท่าทางย่อยที่ไม่ถูกต้องจำนวน 20 ท่าทาง จากสมการการคำนวณคะแนนข้างต้น คะแนนจะมีค่าเท่ากับ 50 \% ดังสมการ
    \begin{equation}
        score_{exercise pose} = \frac{correct subpose}{all subpose}
    \end{equation}
    \begin{equation}
        score_{exercise pose} = \frac{20}{40}
    \end{equation}        
    \begin{equation}
        score_{exercise pose} = 50 \%
    \end{equation}
\end{enumerate}