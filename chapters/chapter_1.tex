\chapter{บทนำ}
\section{ความสำคัญและที่มาของโครงงาน}
\indent
ในปัจจุบัน สังคมมีความสนใจในเรื่องของสุขภาพมากขึ้น การมีสุขภาพที่ดีนั้นเกิดจากการพักผ่อนให้เพียงพอ 
ทานอาหารให้ครบ 5 หมู่ และออกกำลังกายอย่างสม่ำเสมอ  โดยการออกกำลังกายที่ดีนั้นควรที่จะออกกำลังกายได้อย่างถูกวิธี เพื่อลดอาการบาดเจ็บที่อาจเกิดขึ้นในระหว่างการออกกำลังกาย 
และควรออกกำลังกายให้เหมาะสมกับแต่ละบุคคล เพื่อไม่เกิดการหักโหมจนเกินไป นำไปสู่การเจ็บป่วยและหมดกำลังใจในการออกกำลังกาย  
และในสถานการณ์ปัจจุบันที่มีการแพร่ระบาดของไวรัสโคโรนา 2019 (COVID-19) ทำให้การเดินทางไปออกกำลังกายนอกสถานที่ เช่น ฟิตเนส มีความไม่สะดวก อีกทั้งการออกกำลังกายที่ฟิตเนสมีค่าใช้จ่ายที่สูง
\\\indent
ผู้จัดทำได้เห็นความสำคัญของการออกกำลังกาย จึงได้จัดทำโครงงานแอปพลิเคชันสอนออกกำลังกายที่สามารถช่วยจัดท่าทางได้อย่างถูกวิธีเพื่อช่วยให้การออกกำลังกายเป็นเรื่องที่ง่ายสำหรับทุกคน  
และการออกกำลังกายผ่านแอปพลิเคชันฯ ดังกล่าวทำให้เสียค่าใช้จ่ายน้อยลงมากเมื่อเทียบกับค่าใช้จ่ายของฟิตเนส และสามารถออกกำลังกายที่ใดก็ได้

\section{วัตถุประสงค์ของโครงงาน}
\begin{enumerate}
    \item เพื่อส่งเสริมให้ผู้ใช้ได้ออกกำลังกายอย่างถูกวิธี เพื่อลดอาการบาดเจ็บจากการออกกำลังกาย
    \item เพื่อให้ผู้ใช้ได้เห็นถึงความสำคัญของการออกกำลังกาย และช่วยให้การออกกำลังกายเป็นเรื่องง่ายสำหรับผู้ใช้
    \item เพื่อศึกษาและพัฒนาระบบตรวจจับท่าทางของผู้ใช้
    \item เพื่อศึกษาและพัฒนาแอปพลิเคชันสำหรับใช้งานบนสมาร์ตโฟน
\end{enumerate}

\section{ขอบเขตของโครงงาน}
แอปพลิเคชันสอนออกกำลังกายที่พัฒนาขึ้นในโครงงานนี้สามารถช่วยจัดท่าทางการออกกำลังกายให้ผู้ใช้ได้อย่างถูกวิธีนั้นจะประยุกต์ใช้หลักการตรวจจับท่าทาง 
เพื่อนำไปใช้ในการสอนออกกำลังกายให้แก่ผู้ใช้ โดยแอปพลิเคชันสามารถวิเคราะห์ได้ว่าผู้ใช้ได้จัดท่าทางได้ถูกต้องหรือไม่ พร้อมนำเสนอแนวทางในการจัดท่าทางการออกกำลังกายให้ถูกวิธีแก่ผู้ใช้ 
โดยการพัฒนาแอปพลิเคชันที่พัฒนาขึ้นได้กำหนดกลุ่มผู้ใช้หลักที่อายุระหว่าง 15 ถึง 40 ปี และได้กำหนดข้อจำกัดทางด้านเทคนิคของโครงงานไว้ดังนี้
\begin{enumerate}
    \item จะเน้นไปที่ท่าทางการออกกำลังกายแบบพื้นฐาน และโยคะพื้นฐาน
    \item กล้องหน้าโทรศัพท์ ที่ความละเอียดมากกว่าหรือเท่ากับ 7 ล้านพิกเซล
    \item แอปพลิเคชันจะสามารถใช้งานได้ทั้งในระบบ iOS และ Android ซึ่งจะมีความต้องการระบบปฏิบัติการเวอร์ชัน 14 สำหรับระบบ iOS และเวอร์ชัน 11 สำหรับระบบ Android
\end{enumerate}

\section{ประโยชน์ที่คาดว่าจะได้รับ}
\begin{enumerate}
    \item เพื่อให้ผู้ใช้หันมาใส่ใจสุขภาพและออกกำลังกายกันมากขึ้น
    \item เพื่อเสริมสร้างพฤติกรรมการออกกำลังกายให้กับผู้ใช้อย่างถูกต้องและสม่ำเสมอ
\end{enumerate}

\section{แผนการดำเนินงาน}
\noindent
\begin{table}
    \caption{แผนการดำเนินงาน}
    \begin{tabularx}{\textwidth}{ | >{\raggedright} X | c | c | c | c | c | c | c | c | c | }
        \hline
        \bf\centering รายการ & \bf ส.ค. & \bf ก.ย. & \bf ต.ค. & \bf พ.ย. & \bf ธ.ค. & \bf ม.ค. & \bf ก.พ. & \bf มี.ค. & \bf เม.ย.\\
        \hline
        1) กำหนดขอบเขต เป้าหมาย และวัตถุประสงค์ของโครงงาน & \cellcolor[gray]{.7} & \null & \null & \null & \null & \null & \null & \null & \null\\
        \hline
        2) เก็บข้อมูลและวิเคราะห์ความต้องการของกลุ่มเป้าหมาย & \cellcolor[gray]{.7} & \null & \null & \null & \null & \null & \null & \null & \null\\
        \hline
        3) ศึกษาข้อมูล เครื่องมือ และไลบรารีที่เกี่ยวข้อง & \cellcolor[gray]{.7} & \cellcolor[gray]{.7} & \null & \null & \null & \null & \null & \null & \null\\
        \hline
        4) ออกแบบการทำงานของระบบ & \null & \cellcolor[gray]{.7} & \null & \null & \null & \null & \null & \null & \null\\
        \hline
        5) ออกแบบส่วนติดต่อกับผู้ใช้ (User Interface) & \null & \cellcolor[gray]{.7} & \cellcolor[gray]{.7} & \null & \null & \null & \null & \null & \null\\
        \hline
        6) พัฒนาระบบ API & \null & \cellcolor[gray]{.7} & \cellcolor[gray]{.7} & \null & \null & \null & \null & \null & \null\\
        \hline
        7) พัฒนาระบบฐานข้อมูล & \null & \cellcolor[gray]{.7} & \cellcolor[gray]{.7} & \null & \null & \null & \null & \null & \null\\
        \hline
        8) พัฒนาส่วน Frontend ของ Mobile Application & \null & \cellcolor[gray]{.7} & \cellcolor[gray]{.7} & \cellcolor[gray]{.7} & \null & \null & \null & \null & \null\\
        \hline
        9) พัฒนาระบบการตรวจจับท่าทาง และระบบการสอนออกกำลังกาย & \null & \cellcolor[gray]{.7} & \cellcolor[gray]{.7} & \cellcolor[gray]{.7} & \null & \null & \null & \null & \null\\
        \hline
        10) ทดสอบระบบ และเผยแพร่แอปพลิเคชัน & \null & \null & \null & \null & \cellcolor[gray]{.7} & \null & \cellcolor[gray]{.7} & \null & \null\\
        \hline
        11) ปรับแก้ระบบให้สมบูรณ์ & \null & \null & \null & \null & \null & \cellcolor[gray]{.7} & \cellcolor[gray]{.7} & \cellcolor[gray]{.7} & \null\\
        \hline
        12) สรุปผล และจัดทำเล่มโครงงาน & \null & \null & \null & \null & \null & \null & \null & \cellcolor[gray]{.7} & \cellcolor[gray]{.7}\\
        \hline
    \end{tabularx}
\end{table}

