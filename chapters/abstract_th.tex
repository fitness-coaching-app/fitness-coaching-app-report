ในปัจจุบัน สังคมมีความสนใจในเรื่องของสุขภาพมากขึ้น และหนึ่งในกิจกรรมที่จะทำให้มีสุขภาพที่ดีคือการออกกำลังกายอย่างสม่ำเสมอและถูกวิธี และควรออกกำลังกายให้เหมาะสมกับแต่ละบุคคล เพื่อไม่เกิดการหักโหมจนเกินไป นำไปสู่การเจ็บป่วยและหมดกำลังใจในการออกกำลังกาย  และในสถานการณ์ปัจจุบันที่มีการแพร่ระบาดของไวรัสโคโรนา 2019 (COVID-19) ทำให้การเดินทางไปออกกำลังกายนอกสถานที่ เช่น ฟิตเนส มีความไม่สะดวก อีกทั้งการออกกำลังกายที่ฟิตเนสมีค่าใช้จ่ายที่สูง ทางผู้จัดทำได้เห็นความสำคัญของการออกกำลังกาย จึงได้จัดทำโครงงานแอปพลิเคชันสอนออกกำลังกายที่สามารถช่วยจัดท่าทางได้อย่างถูกวิธีขึ้น ซึ่งได้นำเทคโนโลยีการตรวจจับท่าทางร่างกายมาใช้ซึ่งจะสามารถใช้กล้องหน้าของโทรศัพท์ในการตรวจจับได้ และจะมีการประมวลผลท่าทางและแนะนำเมื่อผู้ใช้ทำท่าทางไม่ถูกต้อง และจะมีฟังก์ชันที่จะช่วยกระตุ้นผู้ใช้ในการออกกำลังกายมากขึ้น เช่น การดูตารางคะแนนลีดเดอร์บอร์ดของผู้ใช้ในระบบ, การดูกิจกรรมของผู้ใช้อื่นในระบบ และการให้เหรียญรางวัลเสมือนแก่ผู้ใช้ ซึ่งแอปพลิเคชันนี้ช่วยให้การออกกำลังกายเป็นเรื่องที่ง่ายสำหรับทุกคน  และการออกกำลังกายผ่านแอปพลิเคชันดังกล่าวทำให้เสียค่าใช้จ่ายน้อยลงมากเมื่อเทียบกับค่าใช้จ่ายของฟิตเนส และสามารถออกกำลังกายที่ใดก็ได้